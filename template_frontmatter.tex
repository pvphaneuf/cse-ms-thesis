%
%
% UCSD Doctoral Dissertation Template
% -----------------------------------
% http://ucsd-thesis.googlecode.com
%
%


%% REQUIRED FIELDS -- Replace with the values appropriate to you

% No symbols, formulas, superscripts, or Greek letters are allowed
% in your title.
\title{ALE Analytics: A Software Pipeline and Web Platform for the Analysis of Microbial Genomic Data from Adaptive Laboratory Evolution Experiments}

\author{Patrick Phaneuf}
\degreeyear{\the\year}
\campus{San Diego}  % used as: University of California, \campus


% Master's Degree theses will NOT be formatted properly with this file.
\degreetitle{\thedegree}

\field{Computer Science and Engineering}
% \specialization{Anthropogeny}  % If you have a specialization, add it here

\chair{Professor Bernhard Palsson}
% Uncomment the next line iff you have a Co-Chair
% \cochair{Professor Cochair Semimaster}
%
% Or, uncomment the next line iff you have two equal Co-Chairs.
%\cochairs{Professor Chair Masterish}{Professor Chair Masterish}

%  The rest of the committee members  must be alphabetized by last name.
\othermembers{
Professor Vineet Bafna\\
Professor Pavel Pevzner\\
}
 \ifnum\pdfstrcmp{\@degree}{masters}=0 %
    \numberofmembers{3} % |chair| + |cochair| + |othermembers|
\else
    \numberofmembers{4} % |chair| + |cochair| + |othermembers|
\fi

%% START THE FRONTMATTER
%
\begin{frontmatter}

%% TITLE PAGES
%
%  This command generates the title, copyright, and signature pages.
%
\makefrontmatter


%% SETUP THE TABLE OF CONTENTS
%
\tableofcontents
\listoffigures  % Comment if you don't have any figures
\listoftables   % Comment if you don't have any tables


%% ABSTRACT
%
%  Doctoral dissertation / thesis abstracts should not exceed 350 words.
%   The abstract may continue to a second page if necessary. Masters thesis abstract should not exceed 250 words according to OGS.
%
\begin{abstract}
%Adaptive Laboratory Evolution (ALE) is a tool for studying biological molecular mechanisms and evolutionary dynamics through coupling with whole genome sequencing. Academic and industrial labs involved in the study of microbial evolution and metabolic engineering use ALE methodologies in their research for exploring adaptive mutations \cite{ASM:/content/journal/microbe/10.1128/microbe.6.69.1}. These labs build custom post-processing and computational pipelines for the analysis of the ALE genomic output, though ALE methodologies often include the same fundamental steps to interrogate the evolutionary trajectory of the organisms under study. This thesis will detail the strategies and computational tools used in the ALE project pipeline executed by UCSD's Systems Biology Research Group (SBRG) and describe a system for compiling and analyzing ALE results. This thesis can contribute to the domain of ALE methodologies by detailing an approach to the computational analysis of ALE experiments, will propose a method for hypothesizing key mutations and will explore global trends for all ALE experiment data accumulated. These computational tools and protocols to identifying key mutations and global trends will be leveraged, implemented and evaluated in our ALE Analytics software pipeline and web platform. We therefore hypothesize that we can leverage existing ALE foundational technologies and methodologies to develop a bioinformatics system that will accurately predict the key mutations of ALE experiments and provide context on global mutational trends and correlated experimental conditions.
Adaptive Laboratory Evolution (ALE) methodologies are used for studying microbial adaptive mutations that optimize host metabolism. The Systems Biology Research Group (SBRG) at the University of  California, San Diego, has implemented high-throughput ALE experiment automation that enables the group to expand their experimental evolutions to scales previously infeasible with manual workflows. The data generated by the high-throughput automation now requires a post-processing, content management and analysis framework that can operate on the same scale. We developed a software system which solves SBRG's specific ALE big data to knowledge challenges. The software system is comprised of a post-processing software protocol for quality control and feedback, a software framework and database for data consolidation and a web platform for reports and automated analysis. The automated analysis is evaluated against published ALE experiment results and maintains an average recall of 89.6\%, an average precision of 71.2\% and identifies key mutations in genes \textit{wecA} and \textit{yjiT} not included in the published work. The consolidation of all ALE experiments into a unified resource has enabled the development of ALE Analytics features that compare key mutations across multiple experiments. These features find the genomic regions \textit{rph}, \textit{hns-tdk}, \textit{rpoB}, \textit{rpoC} and \textit{pykF} mutated in more than one ALE experiment published by SBRG. We reason that leveraging this software system relieves the bottleneck in ALE experiment analysis and generates new data mining opportunities for research in understanding system-level mechanisms that govern adaptive evolution.
\end{abstract}


\end{frontmatter}
